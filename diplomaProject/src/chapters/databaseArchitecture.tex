\chapter{Database Architecture}
\label{chapter:database-architecture}
In this chapter we present our database, how we represented our entities, why we used this representation and what database operations we require.

The system uses Hbase\footnote{http://hbase.apache.org/} as it’s default database for storing data, but can be easily integrated with any kind of database by passing a java class for working with the database entities.

HBase is an open source, non-relational, distributed database modeled after Google's BigTable and written in Java. It is developed as part of Apache Software Foundation's Apache Hadoop project and runs on top of HDFS (Hadoop Distributed Filesystem), providing BigTable-like capabilities for Hadoop. That is, it provides a fault-tolerant way of storing large quantities of sparse data (small amounts of information caught within a large collection of empty or unimportant data, such as finding the 50 largest items in a group of 2 billion records).

Hbase is divided into tables. Each table has a name. Then, each table has multiple row keys. Each row key can have multiple families (columns). Each family can have multiple qualifiers. Each qualifier has a value.

\begin{table}[h]
  \caption{Hbase database format}
\centering
\begin{tabular}{@{}lllllll@{}}

Row key                & Family                         &                                & Family                         &                                & Family    &           \\ \toprule 
& Qualifier                      & Qualifier                      & Qualifier                      & Qualifier                      & Qualifier & Qualifier \\ 
                       & Value                          & Value                          & Value                          & Value                          & Value     & Value     
\end{tabular}
\end{table}

Due to HBase's table format, we get fast access to our article data. Also, we modeled our entities so that we can take full advantage of the NOsql database.

\section{Database Overview}
\label{sec:users}
As we can see in Figure 4.1 we have 3 database tabels: Users, Items and TFIDF. The Users table represents an user in the database and it's required data for being able to make personalised recommendations. The Items table represents an item in our database and it's required data for being able to apply TFIDF to it and make item related recommendations. The TFIDF table holds the required data for applying TFIDF on our items.
Each user has a list of items recommended by the user and an item history, referenced by the item's id. Each item has a set of words in it's content and a list of ratings from the users. Each word is saved with the total number of appearences in all the files, in the TFIDF table. This table also keeps track of the total number of files in the database.
\begin{figure}[h]
\caption{Database Schema}
\includegraphics[width=1.0\textwidth]{src/img/databaseSchema.png}
\end{figure}

\section{Database entities}
\label{sec:database-entities}

We have mapped our database over the attributes that are needed for the recommendation and related algorithms and we present them in the following sections. We will begin with the Users table and end with the Items table, since the TFIDF table isn't a basic required table in order to represent our entities.

\subsection{Users}
\label{sec:users}
These are the basic required attributes in order to represent an user entity.
\begin{itemize}
	\item user id - String
		\\ We need an user id to uniquely identify each existing user.
	\item user preferred categories - array of strings
		\\ The user may set certain categories as preferred. 
		The recommendation system will give a greater importance to those categories when making recommendations.
\end{itemize}

\subsection{Items}
\label{sec:items}
These are the basic required attributes in order to represent an item entity
\begin{itemize}
	\item item id - String
		\\ We need an unique identifier for each item
	\item content url - String
		\\ This is an URL or a path which leads to content or information about that item.
		We need it, in order to be able to access the data of that item and provide recomendations based on similarity. 
	\item date created - date
		\\ This is the date at which the article has been created.
		We need it in order to make recommendations based on article similarity.
	\item title - String
		\\ This is the title of an item and it can be even a short desciption.
		We use this in order to make recommendations based on article similarity.
	\item short title - String
		\\ This is the short title of an item.
		We use this in order to make recommendations based on article similarity.
	\item keywords - array of strings
		\\ This is a set of keywords, set by the item's publisher that represent the item.
		We use this in order to make recommendations based on article similarity.
	\item department - String
		\\ This is the department to which an item belongs. It is a more precise term than category. An example of department would be "computer science".
		We use this in order to make recommendations based on article similarity.
	\item category - String
		\\ This is the category to which an  item belongs. It is a more broad term than department. An example of category would be "university"
		We use this in order to make recommendations based on article similarity.
	\item collection references - array of collection URLs
		\\ This is a list of collections to which this item belongs. The publisher may add the item to various collections in order to better organise them and improve the recommendations.
		We use this in order to make recommendations based on article similarity.
	\item author - String
		\\ This is the name of the creator of an item.
		We use this in order to make recommendations based on article similarity.
\end{itemize}

\section{Database Design}
\label{sec:database-design}
Besides the basic information, we need to store some extra data in order to speed up the processing time of the recommended and related articles or improve them.

\subsection{User Table}
\label{sec:users-table}
We need an user table in order to be able to make personalised recommendations for each user.

\begin{table}[h]
\centering
\caption{User table format}
\label{user-table-format}
\begin{tabular}{@{}lllll@{}}
Row key & & & & \\ \toprule
User id & Preferred categories & Top friends & Item history & Items recommended directly  \\ 
        & Preferred categories & Top friends & Item history & Items recommended directly \\ 
        & array list of categories & array list of user ids & array  list of item ids & array list of items\\ 
\end{tabular}
\end{table}

\begin{itemize}
	\item item history - array of item ids
		\\ We need this in order to find the similarity between two users and make collaboative filtering recommendations.
		The item history may also be used to make recommendations for friends
	\item items recommended for the user - array of item ids (caching)
		\\ In order to speed up things we cache the items recommended for a user.	
	\item top friends - array of user ids
		\\ The user may add friends to their account. If no friends are selected then a list will be automatically generated by calculating the similarity between different users.
		We use the friend list in order to make recommendations based on the friend's recommended articles, item history and items recommended directly by that user. 
	\item items recommended directly by the user - array of item ids
		\\ These items have been directly recommended by the user and will be recommended to friends of this user.
		These items can also be used when doing collaborative filtering, since we know that the user prefers them.
\end{itemize}

\subsection{Item Table}
\label{sec:item-table}
We need this table in order to be able to represent an item in the database and not use another system to save them.
\begin{table}[h]
\centering
\caption{Item table format}
\label{item-table-format}
\begin{tabular}{@{}ccccllll@{}}
Row key &  &  &  &  &  &  &  \\ \toprule
Item id & \multicolumn{1}{c}{Content URL} & \multicolumn{1}{c}{Date created} & \multicolumn{1}{l}{Title} & \multicolumn{1}{l}{Short Title} & \multicolumn{1}{l}{Keywords} & \multicolumn{1}{l}{Department} \\ 
Item id & \multicolumn{1}{c}{Content URL} & \multicolumn{1}{c}{Date created} & \multicolumn{1}{l}{Title} & \multicolumn{1}{l}{Short Title} & \multicolumn{1}{l}{Keywords} & \multicolumn{1}{l}{Department} \\ 
String & \multicolumn{1}{c}{String} & \multicolumn{1}{c}{Long} & \multicolumn{1}{l}{String} & \multicolumn{1}{l}{String} & \multicolumn{1}{l}{Array of String} & \multicolumn{1}{l}{String} \\ 
\end{tabular}
\end{table}

\begin{table}[h]
\centering
\label{item-tabel-format-two}
\begin{tabular}{@{}cllllll@{}}
Row key &  &  &  &  &  &  \\ \toprule
Item id & \multicolumn{1}{l}{Category} & \multicolumn{1}{l}{Collection references} & \multicolumn{1}{l}{Author} & \multicolumn{1}{l}{Ratings} & \multicolumn{1}{l}{} & \multicolumn{1}{l}{} \\ 
Item id & \multicolumn{1}{l}{Category} & \multicolumn{1}{l}{Collection references} & \multicolumn{1}{l}{Author} & \multicolumn{1}{l}{User id} & \multicolumn{1}{l}{User id} & \multicolumn{1}{l}{User id} \\ 
String & \multicolumn{1}{l}{String} & \multicolumn{1}{l}{Array of collection URLs} & \multicolumn{1}{l}{String} & \multicolumn{1}{l}{Double} & \multicolumn{1}{l}{Double} & \multicolumn{1}{l}{Double} \\ 
\end{tabular}
\end{table}

\begin{table}[!h]
\centering
\label{item-tabel-format-three}
\begin{tabular}{@{}cllll@{}}
Row key &  &  &  &  \\ \toprule
Item id & \multicolumn{1}{l}{TFIDF} & \multicolumn{1}{l}{} & \multicolumn{1}{l}{} & \multicolumn{1}{l}{Content} \\ 
Item id & \multicolumn{1}{l}{Word} & \multicolumn{1}{l}{Word} & \multicolumn{1}{l}{Word} & \multicolumn{1}{l}{Content} \\ 
String & \multicolumn{1}{l}{Double} & \multicolumn{1}{l}{Double} & \multicolumn{1}{l}{Double} & \multicolumn{1}{l}{String} \\ 
\end{tabular}
\end{table}

\begin{itemize}
	\item rating - map of userId, double
		\\ These are the ratings given by users to this item.
		We use this in order to make recommendations based on collaborative filtering to users.
		Users that gave simillar ratings to the same item most likely have common interests and preferences.
	\item TF - map of word, double
		\\ This will store the term frequency value of the top 100 words for an item.
		We need this in order to speed up the recommandations based on item similarity.
		This value will be updated after a certain time, since we need the IDF(inverse document frequency) of all the words existing in the database in order to calculate the top 100 words.
	\item Content - String
		\\ If a content URL can't be provided, then the content can be saved directly, as a string. 
\end{itemize}

\subsection{TFIDF Table}
\label{sec:tfidf-table}

\begin{table}[!htbp]
\centering
\caption{TFIDF table }
\label{TFIDF-table}
\begin{tabular}{lllll}

\multicolumn{1}{l}{Row key} & \multicolumn{1}{l}{Row key} & \multicolumn{1}{l}{} & \multicolumn{1}{l}{} & \multicolumn{1}{l}{} \\ \toprule
Total file appearences        & Item name                    &                       & Item id               &                       \\
Total file appearences        & Word                         & Word                  & Word                  & Word                  \\
Integer                       & Integer                      & Integer               & Integer               & Integer              
\end{tabular}
\end{table}

\begin{itemize}
	\item Total file appearences - Integer
	\\ We need this in order to keep track of the total number of files existing in the database and to be able to compute the TFIDF
	\item Item name, Word, Integer
	\\ We need this in order to keep track of the total number of times an word appears in all the files.
	If the word is used very often then, it will have a lesser importance in the computation of the TFIDF.
\end{itemize}

\section{Database Operations}
\label{sec:database-operations}
In order to use the database's entities we need a couple of database operations.
If we want to use the system with another type of database, besides Hbase, these operations will have to be implemmented for that specific api.

\begin{itemize}
	\item Check if a table exists
	\\ We need this in order to check if a table exists, before doing any operations on it.
	\item Create a table
	\\ If the database is new and the table does not exist we need to create it.
	\item Delete a table
	\\ We need this in order to clear the database and delete all tables, if needed
	\item Add item to a table
	\\ We need this in order to add entities and data to those entities.
	\item Delete item from a table
	\\ We need this in order to remove items from the table when they are no longer needed
	\item Get one item from a table
	\\ We need this in order to access the database's data.
\end{itemize}

